\section*{ВВЕДЕНИЕ}
\addcontentsline{toc}{section}{ВВЕДЕНИЕ}

\par За последнии 10 лет рынок аудиостриминга значительно вырос и претерпел сильные изменения\cite{robeco}. 
Всё больше людей пользуются сервисами потокового прослушивания музыки, подкастов и аудиокниг.
Иследование \cite{scirp} показало, что число пользователей мобильных устройств превысило $2.5$ миллиарда человек.
Ежегодный прирост в $5-10\%$ пользователей способствет к стремительному увеличению популярности мобильных сервисов.
В результате возникает потребность в поддержке воспроизведения потокововых аудиоданных на мобильных устройствах.
  

\par Цель работы – провести классификацию методов воспроизведения потокового аудио в мобильном приложении на операционной системе iOS.
\par Для достижения поставленной цели в ходе работы требуется решить следующие задачи:

\begin{itemize}
    \item[---] рассмотреть цифровое представление аудиоданных;
    \item[---] провести анализ форматов хранения аудиоданных;
    \item[---] провести анализ протоколов передачи потоковых данных;
    \item[---] провести анализ существующих средств воспроизведения аудиоданных в операционной системе iOS;
\end{itemize} 

