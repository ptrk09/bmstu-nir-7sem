\setupsectionstar
\section*{ЗАКЛЮЧЕНИЕ}
\addcontentsline{toc}{section}{ЗАКЛЮЧЕНИЕ}
    \par Среди рассмотренных методов воспроизведения потокового аудио в мобильном приложении на операционной системе iOS наиболее подходящим решением, 
    с точки зрения поддерживаемых форматов ауди-файлов (ACC и MP3), протоколов передачи потоковых аудиоданных HLS и RTSP, является метод воспроизведение аудиоданных с помощью системного сегментатора.
    
    В результате выполнения научной исследовательской работы была достигнута поставленная цель --- проведена классификация методов воспроизведения потокового аудио в мобильном приложении на операционной системе iOS.
    Для достижения поставленной цели в ходе работы были решены следующие задачи:
    \begin{itemize}
        \item[---] рассмотрено цифровое представление аудиоданных;
        \item[---] проведён анализ форматов хранения аудиоданных;
        \item[---] проведён анализ протоколов передачи потоковых данных;
        \item[---] проведён анализ существующих средств воспроизведения аудиоданных в операционной системе iOS;
    \end{itemize} 


% \par Среди рассмотренных методов хранения и защиты данных от несанкционированного доступа наиболее защищенным является распределенный реестр, внутри которого в свою очередь выбор между Блокчейном и Holochain зависит от потребности в анонимности в реализуемой системе.
% \par В результате выполнения научной исследовательской работы была достигнута поставленная цель --- исследованы и классифицированы существующие методы хранения и защиты информации от несанкционированного доступа. На пути к достижению цели были решены следующие задачи:

% \begin{itemize}
% \item[---] обозначена проблема и описаны основные понятия предметной области;
% \item[---] проведен анализ существующих методов хранения и защиты данных от несанкционированного доступа.
% \end{itemize} 
\pagebreak